\documentclass[a4paper,11pt]{article}

% packages
\usepackage{latexsym}
\usepackage[pdftex]{hyperref}
\usepackage[usenames,dvipsnames]{color}
\usepackage[cm]{fullpage}
\usepackage{parskip}
%\usepackage[empty]{fullpage}

\hypersetup{
  colorlinks,
  citecolor=black,
  filecolor=black,
  linkcolor=black,
  urlcolor=black
}
\urlstyle{same}

\newenvironment{pitemize}{
\begin{itemize}
\setlength{\itemsep}{.01in}
\setlength{\parskip}{.01in}
}
{\end{itemize}}

\setlength{\parindent}{0in}
\setlength{\parskip}{0in}

\pagestyle{empty}

\begin{document}

\Huge{\textbf{Scott Kruger}}
\hrule
\vspace{0.05in}
\normalsize{310 Avondale Ave \hfill (314) 640-0882 \hfill \href{mailto:scott@chojin.org}{scott@chojin.org} \\
Champaign, IL 61820 \hfill \href{http://chojin.org}{http://chojin.org}}

\section*{\huge{Education}}
\hrule
\vspace{0.1in}
\textbf{UNIVERSITY OF ILLINOIS} \hfill \textit{Urbana-Champaign, IL}
\begin{itemize}
\setlength{\itemsep}{0.1in}
\item[]M.S. Astronomy \hfill \textit{Dec, 2010}
\item[]B.S. Computer Science \hfill \textit{May, 2008} \\
Specialization: Computational Science and Engineering
\end{itemize}


\section*{\huge{Experience}}
\hrule
\vspace{0.1in}
\textbf{WOLFRAM RESEARCH INC} \hfill \textit{Champaign, IL} \\
Technical Services Engineer \hfill \textit{Oct, 2010 - Present}
\begin{pitemize}
\item[-]Solved Mathematica programming queries from users via phone and email
\item[-]Worked with developers to gather information about bugs and investigate workarounds
\item[-]Focused on GPU's, Java/.NET links, C compilation, and cluster/grid computing
\item[-]Developed internal training materials for Java/.NET links and cluster/grid computing
\item[-]Wrote public tutorial and how-to documents for cluster/grid computing and OpenCL/CUDA
\end{pitemize}
\vspace{0.2in}
\textbf{UNIVERSITY OF ILLINOIS} \hfill \textit{Urbana-Champaign, IL} \\
Research Assistant \hfill \textit{May, 2009 - Oct, 2010}
\begin{pitemize}
\item[-]Identified opportunities to exploit parallelism in Discrete Spherical Harmonics Transforms
\item[-]Implemented optimized DSHT routine on GPU using CUDA
\end{pitemize}
Teaching Assistant \hfill \textit{Aug, 2008 - May, 2009}
\begin{pitemize}
\item[-]Taught introductory astronomy course discussion sections
\item[-]Adapted course guidelines to form weekly lesson plan
\end{pitemize}
Undergraduate Assistant \hfill \textit{Dec, 2006 - Aug, 2008}
\begin{pitemize}
\item[-]Created and maintained BOINC distributed computing project, Cosmology@Home
\item[-]Administrated workstations/servers for astronomy research group and physics IT
\item[-]Assembled and maintained 16-node HPC cluster running Scientific Linux
\end{pitemize}

\section*{\huge{Skills}}
\hrule
\vspace{0.1in}
\textbf{Programming Languages} \\
Proficient: C/C++, Clojure, Mathematica, CUDA \\
Familiar: Shell (bash), Ruby, Java, FORTRAN, PHP, Perl, Python, MIPS assembly, Rails, HTML

\vspace{0.1in}

\textbf{Computer Skills} \\
Linux server/workstation administration, computer assembly/maintenance

\vspace{0.1in}

\textbf{Language Skills} \\
English (native), German (proficient)

\section*{\huge{Projects}}
\hrule
\vspace{0.1in}
\textbf{\href{http://github.com/thesquelched/clj-yahoo}{clj-yahoo}}
 - Clojure library for accessing Yahoo API services

\end{document}
